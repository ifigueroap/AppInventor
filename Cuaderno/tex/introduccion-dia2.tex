Hola! Bienvenido al segundo día del taller Programa Tus Ideas :)

\paragraph{}
El día de hoy comenzaremos a trabajar con el componente
\component{Lienzo}, que te permite dibujar líneas y círculos, y
mostrar imágenes y animaciones. En el primer tutorial desarrollarás
una aplicación estilo ``paint'', donde puedes tomar una foto con la
cámara y luego editarla usando líneas, círculos y distintos colores.
Luego, aprenderás a programar animaciones básicas implementado el
juego \appName{Atrapa el Topo}, inspirado por el juego mecánico donde
tienes que darle martillazos a esos pobres animalitos.

\paragraph{}
Luego de una breve discusión, el resto de este documento consiste en
material de apoyo sobre los siguientes conceptos fundamentales: el
componente \component{Lienzo}, cómo programar usando \emph{variables},
y el uso de \emph{temporizadores}. En conjunto, estos componentes de
\AppInventor te permitirán programar juegos bastante sofisticados. Las
actividades relacionadas con juegos y animaciones las continuaremos
mañana, en el Día 3, por lo que es importante que las domines!

