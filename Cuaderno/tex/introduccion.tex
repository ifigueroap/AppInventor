\chapter*{Introducción}


Hola! Bienvenido a tu primer día en el taller de programación Programa Tus Ideas :)

\paragraph{}
En este taller aprenderás a programar aplicaciones móviles para dispositivos Android usando el entorno de desarrollo \AppInventor. \AppInventor fue desarrollado por Google y está diseñado específicamente para poder ser usado por cualquier persona---sí, incluso si no sabes mucho de computación o programación puedes desarrollar tus aplicaciones, sólo se requiere mucha imaginación, creatividad y entusiasmo!

\paragraph{}
Las actividades del taller están distribuidas en 8 días. Cada día te entregaremos un documento como éste, con los tutoriales y material de apoyo que necesitarás para ir aprendiendo cómo programar tus ideas. En el primer día aprenderás a usar el entorno \AppInventor y crearás tus primeras aplicaciones (sí! más de una!). El índice que se presenta a continuación muestra el contenido de este documento: primero, un tutorial paso por paso para tu primera aplicación: \appName{Hola Gatito}. A continuación un breve tutorial te muestra cómo crear tu Portafolio, una página web para mostrar tus aplicaciones al mundo, donde tus amigos y otras personas pueden instalar tus aplicaciones en sus dispositivos. Luego, unas preguntas y ejercicios de programación para consolidar lo aprendido en el tutorial. A continuación un proyecto para que eches a volar tu imaginación. La~\Cref{sec:material-de-apoyo} es opcional, contiene un resumen de toda la ``materia'' de este primer día. Finalmente, un breve resumen con los objetivos y contenidos de este primer día.

\paragraph{}
A medida que vayamos desarrollando las aplicaciones utilizaremos material multimedia: imagenes y sonidos. Todo este material estará disponible en una carpeta compartida de DropBox. En los documentos nos referiremos a esta carpeta como la carpeta \resources{ProgramaTusIdeas}. Consulta con tu tutor por el link específico! Recuerda además que puedes consultar con el tutor del taller ante cualquier problema o consulta. También puedes trabajar desde tu casa en las actividades del taller y podrás seguir usando \AppInventor después de terminado el taller!
